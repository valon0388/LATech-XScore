% Generated by Sphinx.
\documentclass[letterpaper,10pt,english]{manual}
\usepackage[utf8]{inputenc}
\usepackage[T1]{fontenc}
\usepackage{babel}
\usepackage{times}
\usepackage[Bjarne]{fncychap}
\usepackage{longtable}
\usepackage{sphinx}


\title{XScore Documentation}
\date{May 05, 2010}
\release{1.0}
\author{XScorers}
\newcommand{\sphinxlogo}{}
\renewcommand{\releasename}{Release}
\makeindex
\makemodindex
\newcommand\PYGZat{@}
\newcommand\PYGZlb{[}
\newcommand\PYGZrb{]}
\newcommand\PYGaz[1]{\textcolor[rgb]{0.00,0.63,0.00}{#1}}
\newcommand\PYGax[1]{\textcolor[rgb]{0.84,0.33,0.22}{\textbf{#1}}}
\newcommand\PYGay[1]{\textcolor[rgb]{0.00,0.44,0.13}{\textbf{#1}}}
\newcommand\PYGar[1]{\textcolor[rgb]{0.73,0.38,0.84}{#1}}
\newcommand\PYGas[1]{\textcolor[rgb]{0.25,0.44,0.63}{\textit{#1}}}
\newcommand\PYGap[1]{\textcolor[rgb]{0.00,0.44,0.13}{\textbf{#1}}}
\newcommand\PYGaq[1]{\textcolor[rgb]{0.38,0.68,0.84}{#1}}
\newcommand\PYGav[1]{\textcolor[rgb]{0.00,0.44,0.13}{\textbf{#1}}}
\newcommand\PYGaw[1]{\textcolor[rgb]{0.13,0.50,0.31}{#1}}
\newcommand\PYGat[1]{\textcolor[rgb]{0.73,0.38,0.84}{#1}}
\newcommand\PYGau[1]{\textcolor[rgb]{0.32,0.47,0.09}{#1}}
\newcommand\PYGaj[1]{\textcolor[rgb]{0.00,0.44,0.13}{#1}}
\newcommand\PYGak[1]{\textcolor[rgb]{0.14,0.33,0.53}{#1}}
\newcommand\PYGah[1]{\textcolor[rgb]{0.00,0.13,0.44}{\textbf{#1}}}
\newcommand\PYGai[1]{\textcolor[rgb]{0.73,0.38,0.84}{#1}}
\newcommand\PYGan[1]{\textcolor[rgb]{0.13,0.50,0.31}{#1}}
\newcommand\PYGao[1]{\textcolor[rgb]{0.25,0.44,0.63}{\textbf{#1}}}
\newcommand\PYGal[1]{\textcolor[rgb]{0.00,0.44,0.13}{\textbf{#1}}}
\newcommand\PYGam[1]{\textbf{#1}}
\newcommand\PYGab[1]{\textit{#1}}
\newcommand\PYGac[1]{\textcolor[rgb]{0.25,0.44,0.63}{#1}}
\newcommand\PYGaa[1]{\textcolor[rgb]{0.19,0.19,0.19}{#1}}
\newcommand\PYGaf[1]{\textcolor[rgb]{0.25,0.50,0.56}{\textit{#1}}}
\newcommand\PYGag[1]{\textcolor[rgb]{0.13,0.50,0.31}{#1}}
\newcommand\PYGad[1]{\textcolor[rgb]{0.00,0.25,0.82}{#1}}
\newcommand\PYGae[1]{\textcolor[rgb]{0.13,0.50,0.31}{#1}}
\newcommand\PYGaZ[1]{\textcolor[rgb]{0.25,0.44,0.63}{#1}}
\newcommand\PYGbf[1]{\textcolor[rgb]{0.00,0.44,0.13}{#1}}
\newcommand\PYGaX[1]{\textcolor[rgb]{0.25,0.44,0.63}{#1}}
\newcommand\PYGaY[1]{\textcolor[rgb]{0.00,0.44,0.13}{#1}}
\newcommand\PYGbc[1]{\textcolor[rgb]{0.78,0.36,0.04}{#1}}
\newcommand\PYGbb[1]{\textcolor[rgb]{0.00,0.00,0.50}{\textbf{#1}}}
\newcommand\PYGba[1]{\textcolor[rgb]{0.02,0.16,0.45}{\textbf{#1}}}
\newcommand\PYGaR[1]{\textcolor[rgb]{0.25,0.44,0.63}{#1}}
\newcommand\PYGaS[1]{\textcolor[rgb]{0.13,0.50,0.31}{#1}}
\newcommand\PYGaP[1]{\textcolor[rgb]{0.05,0.52,0.71}{\textbf{#1}}}
\newcommand\PYGaQ[1]{\textcolor[rgb]{0.78,0.36,0.04}{\textbf{#1}}}
\newcommand\PYGaV[1]{\textcolor[rgb]{0.25,0.50,0.56}{\textit{#1}}}
\newcommand\PYGaW[1]{\textcolor[rgb]{0.05,0.52,0.71}{\textbf{#1}}}
\newcommand\PYGaT[1]{\textcolor[rgb]{0.73,0.38,0.84}{#1}}
\newcommand\PYGaU[1]{\textcolor[rgb]{0.13,0.50,0.31}{#1}}
\newcommand\PYGaJ[1]{\textcolor[rgb]{0.56,0.13,0.00}{#1}}
\newcommand\PYGaK[1]{\textcolor[rgb]{0.25,0.44,0.63}{#1}}
\newcommand\PYGaH[1]{\textcolor[rgb]{0.50,0.00,0.50}{\textbf{#1}}}
\newcommand\PYGaI[1]{\fcolorbox[rgb]{1.00,0.00,0.00}{1,1,1}{#1}}
\newcommand\PYGaN[1]{\textcolor[rgb]{0.73,0.73,0.73}{#1}}
\newcommand\PYGaO[1]{\textcolor[rgb]{0.00,0.44,0.13}{#1}}
\newcommand\PYGaL[1]{\textcolor[rgb]{0.02,0.16,0.49}{#1}}
\newcommand\PYGaM[1]{\colorbox[rgb]{1.00,0.94,0.94}{\textcolor[rgb]{0.25,0.50,0.56}{#1}}}
\newcommand\PYGaB[1]{\textcolor[rgb]{0.25,0.44,0.63}{#1}}
\newcommand\PYGaC[1]{\textcolor[rgb]{0.33,0.33,0.33}{\textbf{#1}}}
\newcommand\PYGaA[1]{\textcolor[rgb]{0.00,0.44,0.13}{#1}}
\newcommand\PYGaF[1]{\textcolor[rgb]{0.63,0.00,0.00}{#1}}
\newcommand\PYGaG[1]{\textcolor[rgb]{1.00,0.00,0.00}{#1}}
\newcommand\PYGaD[1]{\textcolor[rgb]{0.00,0.44,0.13}{\textbf{#1}}}
\newcommand\PYGaE[1]{\textcolor[rgb]{0.25,0.50,0.56}{\textit{#1}}}
\newcommand\PYGbg[1]{\textcolor[rgb]{0.44,0.63,0.82}{\textit{#1}}}
\newcommand\PYGbe[1]{\textcolor[rgb]{0.40,0.40,0.40}{#1}}
\newcommand\PYGbd[1]{\textcolor[rgb]{0.25,0.44,0.63}{#1}}
\newcommand\PYGbh[1]{\textcolor[rgb]{0.00,0.44,0.13}{\textbf{#1}}}
\begin{document}

\maketitle
\tableofcontents



This is the documentation for XScore, a framework for hosting network security competitions.

\resetcurrentobjects
\hypertarget{--doc-Introduction}{}

\chapter{Introduction}

XScore is the scoring server for Cyber Storm, a network security
competition in which teams will attempt to \emph{hack} each others
networks.  Teams are to provide a number of network services that
they will be expected to maintain despite attacks from opposing teams.
The effectiveness of these attacks will be the determining factor in
deciding the winner of the competition.

The XScore system provides the following services:
\begin{itemize}
\item {} 
Checking the status of various network services provided by teams.

\item {} 
Maintaining a running score of throughout the competition.

\item {} 
Providing an administrative interface to manage competition related events.

\end{itemize}


\section{Goal}

The goal of this project is to provide a framework for not only Cyber Storm, but also
for future network security competitions.  It is hoped that the infrastructure provided
by XScore will be useful and provide a solid foundation upon which Louisiana Tech, as well
as other institutions, can use to host similiar competitions


\section{License}

This program is free software: you can redistribute it and/or modify
it under the terms of the GNU General Public License as published by
the Free Software Foundation; either version 3 of the License, or
(at your option) any later version.

Copyright 2010 Cyber Storm

\resetcurrentobjects
\hypertarget{--doc-Requirements}{}

\chapter{Requirements}

To implement the infrastructure needed for the Cyber Storm competition, XScore
will need to provide two main services: network service checking and a
point based scoring system to rank the opposing teams.  This document
describes the requirements of this system.


\section{Network Service Checking}

Throughout the competition, each team will be expected to maintain a number of
machines that provide a set of network services.  XScore will check the status
of these services on a regular interval to determine if another team has
successfully \emph{hacked} or disabled this service.  Support for checking the
following services will be provided:
\begin{itemize}
\item {} 
HTTP - Hyper Text Transfer Protocol

\item {} 
FTP - File Transfer Protocol

\item {} 
MySQL - Database Server

\item {} 
SSH - Secure SHell

\end{itemize}


\section{Scoring}

To determine a winner of the competition, XScore will need to provide
a system that awards each team points based on the status of the services
they are able to maintain.  XScore is to provide an automated system that
will check these services and then award the given team an appropriate number
of points.

To establish a well-defined method concerning when points are awarded, we will
define the status of a service as one of the following states:
\begin{enumerate}
\item {} 
Service is up and operates as expected.

\item {} 
Service has been successfully hacked by an opposing team.

\item {} 
Service is down or does not operate as expected.

\end{enumerate}

Each of these states will be recognized by XScore and will be the basis on which
points are awarded.  The value of the points can be specified for each of the services
checked and configurable as needed.

In addition to this process of checking services and awarding points, XScore is to
provide the officials of the competition's with the ability to manually adjust
the scores.  This is intended to provide a fallback in case if the scoring server
incorrectly awards a team points.  In addition, this will allow the officials to
handle unforeseen events during the competition.  For example, if a team attempts to
\emph{hack} XScore and modify the scores, officials may want to remove these points.


\section{User Interface to XScore}

To allow officials to easily run the scoring server and manage the compeititoin,
XScore will provide the appropriate software.  Running the scoring server will
be handled by a command-line program and will schedule when and how often services
are checked.  To monitor the status of the competition and provide a friendly user
interface to the scoring server, a web-based application will be included.  This
program can be used by officials to create new challenges and announce important messages.
In addition, this program will provide the officials with access to manually adjust the
scores as needed.

\resetcurrentobjects
\hypertarget{--doc-Design}{}

\chapter{Design}

The XScore system is divided into multiple components in order to provide a modular design
that is easy to extend while also satisfying the neccessary requirements of Cyber Storm.
This document is intended to provide an overview of how XScore is designed and implemented.


\section{Overview}

XScore consists of two main components: the scoring server and a visual front-end.
The scoring server is responsible for scheduling when services are checked and the
registering of the appropriate scoring events.  In addition, the scoring server handles
passing data to the scoring client via an HTTP server.

The visual front-end is a web-based application that provides an
adminstrative interface to the scoring server.  It provides the
ability to monitor the progress of the competition as well manage
various competition related events.


\section{User Interfaces}


\subsection{Scoring Server}

The scoring server is the command-line program \code{xscored} which is used to initialize
and run the server.


\subsection{Front-end}


\section{Programming Environment}

XScore utilizes a number of different software systems to implement the scoring server.
Although primarily written in Python, it also makes use of Bash shell scripts and the front-end
makes extensive use of Javascript.  Since this system is intended to be used on Linux based servers,
little concern was given to portability on non Posix systems.  This allowed us to simplify the
implementation of the scoring server and prevented us from having to test on other platforms.
It does however imply that additional development would be required if this feature were wanted in
the future.

Other systems used and required by XScore include a MySQL database server and a HTTP server used
to communicate with the scoring client and host the front-end.  Refer to the README file included
with the distrubution for the specific versions of the programs used by XScore.


\section{Scoring Server}

The scoring server is divided into four main components:
\begin{enumerate}
\item {} 
Network Service Checkers.

\item {} 
Scheduling of Service Checks.

\item {} 
Scores and Events Handeling.

\item {} 
Communication with the client.

\end{enumerate}

These components are discussed in the following sections.


\subsection{Network Service Checkers}

The network service checkers are used to determine the status of a team's network service.


\subsubsection{FTP}


\subsubsection{HTTP}


\subsubsection{SSH}


\subsubsection{MySQL}


\subsection{Scheduling of Service Checks}


\subsection{Scores and Events Handeling}

Record and manage the scores for each team and the events that occur during the competition.


\subsection{Communication with the Client}

\resetcurrentobjects
\hypertarget{--doc-xscore}{}

\chapter{\texttt{xscore}}

The \code{xscore} package implements the routines used by XScore to check
network services and handle new scoring events.


\section{\texttt{checkers}}
\index{xscore.checkers (module)}
\hypertarget{module-xscore.checkers}{}
\declaremodule[xscore.checkers]{}{xscore.checkers}
\modulesynopsis{}
Module that handles checking the current status of SQL, FTP,
SSH, and HTTP services for each team.
\index{FTP (class in xscore.checkers)}

\hypertarget{xscore.checkers.FTP}{}\begin{classdesc}{FTP}{team, ip, port, usr=None, passwd=None, timeout=30, koth=False}
FTP Service class
\index{get() (FTP method)}

\hypertarget{xscore.checkers.FTP.get}{}\begin{methoddesc}{get}{}
Attempt a connection to FTP server and return welcome message from server.
\end{methoddesc}
\end{classdesc}
\index{HTTP (class in xscore.checkers)}

\hypertarget{xscore.checkers.HTTP}{}\begin{classdesc}{HTTP}{team, ip, port, usr=None, passwd=None, timeout=30, koth=False}
HTTP Service class
\index{get() (HTTP method)}

\hypertarget{xscore.checkers.HTTP.get}{}\begin{methoddesc}{get}{}
Connects to http URL and returns server response.
\end{methoddesc}
\end{classdesc}
\index{MYSQL (class in xscore.checkers)}

\hypertarget{xscore.checkers.MYSQL}{}\begin{classdesc}{MYSQL}{team, ip, port, usr=None, passwd=None, timeout=30, koth=False}
MYSQL Service class
\index{get() (MYSQL method)}

\hypertarget{xscore.checkers.MYSQL.get}{}\begin{methoddesc}{get}{}
Attempt a connection to MYSQL server and return list of all databases.
\end{methoddesc}
\end{classdesc}
\index{SSH (class in xscore.checkers)}

\hypertarget{xscore.checkers.SSH}{}\begin{classdesc}{SSH}{team, ip, port, usr=None, passwd=None, timeout=30, koth=False}
SSH Service class
\index{get() (SSH method)}

\hypertarget{xscore.checkers.SSH.get}{}\begin{methoddesc}{get}{}
Attempt a SSH connection to ip and return MOTD from server.
\end{methoddesc}
\end{classdesc}
\index{Service (class in xscore.checkers)}

\hypertarget{xscore.checkers.Service}{}\begin{classdesc}{Service}{team, ip, port, usr=None, passwd=None, timeout=30, koth=False}
Service class extended by all Network Services, contains facilities
to check status of itself and format a response message.
\index{check() (Service method)}

\hypertarget{xscore.checkers.Service.check}{}\begin{methoddesc}{check}{}
Check the status of this service.

Returns the status and a descriptive message.
\end{methoddesc}
\index{format() (Service method)}

\hypertarget{xscore.checkers.Service.format}{}\begin{methoddesc}{format}{fmt=None}
Construct a fancy message describing this service.

{\color{red}\bfseries{}{}`}fmt' can contain the following \%-style mapping keys which are 
replaced with the appropriate values.
\begin{quote}

\%(team)     Team name
\%(ip)       IP address
\%(port)     Port
\%(service)  The service checked
\%(status)   Status of the service (UP \textbar{} DOWN \textbar{} PWNED)
\%(hacker)   If {\color{red}\bfseries{}{}`}status' is PWNED this contains the hacker's name.
\%(reason)   If {\color{red}\bfseries{}{}`}status' is DOWN this describes why.
\end{quote}
\end{methoddesc}
\index{stat() (Service method)}

\hypertarget{xscore.checkers.Service.stat}{}\begin{methoddesc}{stat}{}
HTTP Service class

Get the status of this service.

Returns `UP', `DOWN', or `PWNED'
\end{methoddesc}
\end{classdesc}
\index{TimeOutException}

\hypertarget{xscore.checkers.TimeOutException}{}\begin{excdesc}{TimeOutException}
Triggered when a service takes too long to respond.
\end{excdesc}
\index{check() (in module xscore.checkers)}

\hypertarget{xscore.checkers.check}{}\begin{funcdesc}{check}{team, service, ip, port, usr='', passwd='', timeout=30}
Checks the status of a service and appropriates points accordingly.
\end{funcdesc}
\index{timeout() (in module xscore.checkers)}

\hypertarget{xscore.checkers.timeout}{}\begin{funcdesc}{timeout}{seconds, func, *args, **kwargs}
Executes the function `func' with the given arguments.  If it
does not return within the time specified by `seconds', a
TimeOutException is raised.
\end{funcdesc}


\bigskip\hrule{}\bigskip



\section{\texttt{scores}}
\index{xscore.scores (module)}
\hypertarget{module-xscore.scores}{}
\declaremodule[xscore.scores]{}{xscore.scores}
\modulesynopsis{}
Database Facilities to handle creation and management of events, 
announcements, and challeneges.
\index{add\_announcement() (in module xscore.scores)}

\hypertarget{xscore.scores.add_announcement}{}\begin{funcdesc}{add\_announcement}{msg, secs\_to\_display=10}
Add a new announcement to the database.
\end{funcdesc}
\index{add\_event() (in module xscore.scores)}

\hypertarget{xscore.scores.add_event}{}\begin{funcdesc}{add\_event}{team, etype, pts, msg}
Submits new event to database.
\end{funcdesc}
\index{end\_challenge() (in module xscore.scores)}

\hypertarget{xscore.scores.end_challenge}{}\begin{funcdesc}{end\_challenge}{id, winner}
Declare a winner for an existing challenge.
\end{funcdesc}
\index{new\_challenge() (in module xscore.scores)}

\hypertarget{xscore.scores.new_challenge}{}\begin{funcdesc}{new\_challenge}{challenge\_name, pts, msg}
Create a new challenge.
\end{funcdesc}
\index{query() (in module xscore.scores)}

\hypertarget{xscore.scores.query}{}\begin{funcdesc}{query}{sql, sql\_args=None}
Executes query and returns resultset.
\end{funcdesc}


\bigskip\hrule{}\bigskip



\section{\texttt{logger}}
\index{xscore.logger (module)}
\hypertarget{module-xscore.logger}{}
\declaremodule[xscore.logger]{}{xscore.logger}
\modulesynopsis{}
Facilities for logging all activity generated by the
XScore Server, including service status checks and
database manipulation.


\bigskip\hrule{}\bigskip



\section{\texttt{config}}
\index{xscore.config (module)}
\hypertarget{module-xscore.config}{}
\declaremodule[xscore.config]{}{xscore.config}
\modulesynopsis{}
Contains all neccissary configuration data 
for each teams servers as well as all neccissary
scoring values for the competition.

\resetcurrentobjects
\hypertarget{--doc-xscored}{}

\chapter{NAME}


\section{SYNOPSIS}


\section{DESCRIPTION}


\section{OPTIONS}


\section{EXAMPLE}


\section{SEE ALSO}

\resetcurrentobjects
\hypertarget{--doc-xscore-check}{}

\chapter{NAME}


\section{SYNOPSIS}


\section{DESCRIPTION}


\section{OPTIONS}


\section{EXAMPLE}


\section{SEE ALSO}

\resetcurrentobjects
\hypertarget{--doc-xscore-new}{}

\chapter{NAME}


\section{SYNOPSIS}


\section{DESCRIPTION}


\section{OPTIONS}


\section{EXAMPLE}


\section{SEE ALSO}


\chapter{Indices and tables}
\begin{itemize}
\item {} 
\emph{Index}

\item {} 
\emph{Module Index}

\item {} 
\emph{Search Page}

\end{itemize}


\renewcommand{\indexname}{Module Index}
\printmodindex
\renewcommand{\indexname}{Index}
\printindex
\end{document}
